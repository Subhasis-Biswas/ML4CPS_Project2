%%%%%%%%%%%%%%%%%%%%%%%%%%%%%%%%%%%%%%%%%
% Journal Article
% LaTeX Template
% Version 2.0 (February 7, 2023)
%
% This template originates from:
% https://www.LaTeXTemplates.com
%
% Author:
% Vel (vel@latextemplates.com)
%
% License:
% CC BY-NC-SA 4.0 (https://creativecommons.org/licenses/by-nc-sa/4.0/)
%
% NOTE: The bibliography needs to be compiled using the biber engine.
%
%%%%%%%%%%%%%%%%%%%%%%%%%%%%%%%%%%%%%%%%%

%----------------------------------------------------------------------------------------
%	PACKAGES AND OTHER DOCUMENT CONFIGURATIONS
%----------------------------------------------------------------------------------------

\documentclass[
	a4paper, % Paper size, use either a4paper or letterpaper
	10pt, % Default font size, can also use 11pt or 12pt, although this is not recommended
	unnumberedsections, % Comment to enable section numbering
	twoside, % Two side traditional mode where headers and footers change between odd and even pages, comment this option to make them fixed
]{LTJournalArticle}

\addbibresource{ref.bib} % BibLaTeX bibliography file


\setcounter{page}{1} % The page number of the first page, set this to a higher number if the article is to be part of an issue or larger work

\usepackage{url}

\usepackage{graphicx}             % For including images
\usepackage{subfigure}
\usepackage{adjustbox}            % For scaling and packing
\usepackage{subcaption}           % For subfigures
\usepackage{xcolor}         % For color names
\usepackage{hyperref}       % For hyperlinks and colored citations

%%%%

\usepackage{amssymb,amsthm,amsmath}
\usepackage{multirow,epsfig, times,multicol}
\usepackage[mathscr]{eucal}
\usepackage{amsfonts}
\usepackage{wrapfig}
\usepackage{amssymb}

\usepackage{ifthen}


\usepackage{titlesec}

\titlespacing*{\section} % Adjusts section spacing
  {0pt}   % Left indent (usually 0pt for no indent)
  {2pt}  % Space above the section title
  {2pt}   % Space below the section title

\titlespacing*{\subsection}
  {0pt}{2pt}{2pt}


%%%%

% Set colors for links and citations
\hypersetup{
    colorlinks=true,        % Enable colored links
    linkcolor=orange,        % Set link color (e.g., section refs)
    citecolor=blue,         % Set citation color
    urlcolor=cyan           % Set URL color
}



\usepackage{listings}


\lstset{
    basicstyle=\ttfamily, % Use typewriter font
    keywordstyle=\color{blue}, % Keywords in blue
    commentstyle=\color{green!50!black}, % Comments in green
    stringstyle=\color{red}, % Strings in red
    backgroundcolor=\color{gray!10}, % Light gray background
    numbers=left, % Line numbers on the left
    numberstyle=\tiny\color{gray}, % Style for line numbers
    frame=single, % Single frame around code
    tabsize=2, % Set tab size
}

\usepackage{enumitem} % Recommended package for list customization

\setlist[itemize]{topsep=1pt, itemsep=0pt, parsep=1pt, partopsep=0pt}
\setlist[enumerate]{topsep=1pt, itemsep=0pt, parsep=1pt, partopsep=0pt}
\setlength{\topsep}{0pt}     % Space before and after the environment
\setlength{\partopsep}{0pt}   % Extra space added before if needed
\setlength{\parskip}{0pt}     % Space between paragraphs inside verbatim (optional)
\graphicspath{{figures/}}


%----------------------------------------------------------------------------------------
%	TITLE SECTION
%----------------------------------------------------------------------------------------

\title{CP 217: Project 2 Report} % Article title, use manual lines breaks (\\) to beautify the layout

% Authors are listed in a comma-separated list with superscript numbers indicating affiliations
% \thanks{} is used for any text that should be placed in a footnote on the first page, such as the corresponding author's email, journal acceptance dates, a copyright/license notice, keywords, etc
\author{%
	Subhasis Biswas\thanks{22571, CDS MTech 2nd Year},  Pradhumn Sharma\thanks{22559, CDS MTech 2nd Year} and Bhookya Raju\thanks{25076, MTech Mobility Engg \& Mech Dept}
}

% Affiliations are output in the \date{} command


%----------------------------------------------------------------------------------------

\begin{document}

% \onecolumn
% \maketitle % Output the title section

\begin{center}
  \textbf{\Large CP 217: Project 2 Report} \\[1em]
  \textbf{Team: Data Diviners} \\[2em]
\end{center}
\vspace{-8pt}
\textbf{Subhasis Biswas} \> (22571, CDS MTech 2nd Year) \\
\textbf{Pradhumn Sharma} \> (22559, CDS MTech 2nd Year) \\
\textbf{Bhookya Raju} \> (25076, MTech Mobility Engg \& Mech Dept) \\

\vspace{-5pt}
\hrule\leavevmode

% \thispagestyle{empty}
% \mbox{} % Empty box to force the page

% \newpage
\vspace{-3pt}

{\fontsize{8}{6}\selectfont \textit{Notes:} \hyperref[tab:merged_community_names_serial]{Table~\ref{tab:merged_community_names_serial}} serves as a lookup table for the suburb name-to-index mapping, whenever the names of the suburbs themselves are too long to mention. \textit{`Spatial Autocorrelation'} and \textit{`SA'} used in text interchangeably. Blue, Orange, Cyan coloured texts are clickable hyperlinks.}

\section{Part A}



\subsection{A.1}

In this section we refer to the defined metrics as $M1$, $M2$ and $M3$.
\begin{itemize}
    \item \textbf{M1}: Geographic Proximity
    \item \textbf{M2}: Service Availability
    \item \textbf{M3}: Population Age Distribution
\end{itemize}

\subsubsection{Metric Specifications}\leavevmode


Here we'll be going through how the similarity metrics were defined and the related information.

\textbf{M1}: Used $Location$ feature from the excel sheets to approximate the location relative to the \textit{Melbourne General Post Office (GPO)}. The feature-value was parsed as: \newline
\textit{Xkm Y of Melbourne} $\rightarrow (X, Y) \rightarrow (X, compass\ bearing)$, with $Y \in \{\text{N}, \text{NNE}, \text{NE}, \text{ENE}, \dots,\text{NNW}\}$ and the corresponding bearings being in $\{0, 22.5, 45, 67.5,\dots, 337.5\}$.

Now, max distance of any suburb from our origin $O$ (Melbourne GPO) is 68km. So, maximum pairwise distance that can be is $2\times 68=136$ km. The curvature of the Earth causes a drop of around $1.45km$ [\href{https://earthcurvature.com/}{Calculator}] across this distance. As this is quite small, it's ignored and the entire region is treated as flat. We then convert $Polar$ to $Cartesian$ using the formula 

$(x, y) = \big(r\sin\left(\frac{\pi\theta}{180}\right), r\cos\left(\frac{\pi\theta}{180}\right)\big)$ [$y$ axis aligns with $0^\circ$, so $\sin$ and $\cos$ are swapped]. The goal is to check for physical proximity of the suburbs, using the $L^2$ norm.

\textbf{M2}: Each feature under the $Services$ catgory is considered and min-max scaled. Afterwards, mean was taken across the scaled feature-values under the Services category suburb by suburb, which is named as the $Services \ Score$ for that suburb. Similarity between suburbs from this perspective is defined as the absolute difference between the corresponding services' score. The aim is to quantify how well-equipped a suburb is in terms of the said facilities and to check for similar amount of services provided in other suburbs.

\textbf{M3}: The \textit{2012 population} category was chosen and only the feature names with $\%$ were chosen, for example

$\{ \textit{2012 ERP age 0-4\%}, \textit{2012 ERP age 5-9\%}, \dots\}$. The feature values are then converted to probability distributions, with each row summing up to 1 for each suburb (verified). For quantifying similarity between two locations we consider the \href{https://en.wikipedia.org/wiki/Jensen%E2%80%93Shannon_divergence}{Jensen Shannon Divergence}(JSD) between the two population distributions. This distance metric was chosen as we're essentially dealing with pmf's and because JSD is symmetric. The aim of this metric is to measure the similarity between the 2012 (latest) population distributions between two locations.



\subsection{A.2}

We are using the following module for computing the multidimensional scaling onto 2D.
\begin{verbatim}
    from sklearn.manifold import MDS
\end{verbatim}

We'll be using \href{https://en.wikipedia.org/wiki/Procrustes_analysis}{Procrustes Disparity} for measuring the alignment between two distance metrics, the comparisons might be of the sort: \textit{Metric in original space vs Metric in 2D} or \textit{Metric A vs Metric B} etc. The closer the value of the disparity is to 0 the better. For MDS, we'll also note the \href{https://imaging.mrc-cbu.cam.ac.uk/statswiki/FAQ/mds/stress#:~:text=The%20measure%20of%20goodness%20of,or%20more%20estimated%20stimuli%20dimensions}{stress of the MDS} and the correlation between the original pairwise distances and distances in the 2D projected space. The values of the MDS quality metrics are written within the plots.

\textbf{M1}: The suburb locations are already on the 2D map, thus we do not require MDS.  \hyperref[sub@fig:jd_suburb_dist]{Figure~\ref{fig:jd_suburb_dist}} \hyperref[sub@fig:scatter_suburb]{Figure~\ref{fig:scatter_suburb}}

\textbf{M2}: 2D projected profile: \hyperref[sub@fig:mds_services]{Figure~\ref{fig:mds_services}}

\textbf{M3}: 2D projected profile: \hyperref[sub@fig:mds_population]{Figure~\ref{fig:mds_population}}




\subsubsection{Findings}

While doing MDS, for the service score metric, we're essentially dealing with scalar outputs, and trying to put them in 2D. Thus the distance has been well preserved between the original and the 2D space as indicated by the low stress and disparity as well as high correlation. This was expected.

However, for JSD based population distribution similarity, the high amount of stress indicates poor projection onto the 2D. Although it's not exactly a surprise but still that's quite a high amount. Despite of that, the disparity is still low and correlation is high. 


To check which suburbs remain unchanged between these two metrics \textit{M2} and \textit{M3} we do a \href{https://en.wikipedia.org/wiki/Spectral_clustering}{spectral clustering} with the exact same similarity matrices arising out of the original computation. Then we use a Gaussian Kernel with $\sigma=1$. We check for the `elbow' in the (sorted in descending order) eigenvalue plots for both M2 and M3 for the affinity matrices and decide upon the number of clusters required. We settle on $k=3$. The agreement between the two clusterings:

$ARI=0.024, NMI=0.133$;  \hyperref[sub@fig:spectral_clustering]{Figure~\ref{fig:spectral_clustering}}


\subsection{A.3}


The issue with geospatial data analysis is that the observations of the neighbouring places are not independent by the \href{https://en.wikipedia.org/wiki/Tobler%27s_first_law_of_geography}{Tobler's first law of geography}, thus many of the ML algorithms that assume independence of the samples cannot be used here directly. Here, in order to address the \href{https://www.sciencedirect.com/topics/mathematics/spatial-autocorrelation}{Spatial Autocorrelation} we use \href{https://en.wikipedia.org/wiki/Moran%27s_I}{Moran's I} (with permutations set to 9999, random seed = 42) for testing the influence of the geographic proximity. More formally, for Moran's I hypothesis test we set the null hypothesis $H_0$ as:

$H_0:$ \textit{`There is no spatial autocorrelation between the observations'} with significance level set to $\alpha =0.05$. Weight matrix was constructed using KNN with $k=8$ (\href{https://pro.arcgis.com/en/pro-app/latest/tool-reference/spatial-statistics/h-how-spatial-autocorrelation-moran-s-i-spatial-st.htm}{Recommendation}) due to the varying spatial density of the suburbs.

We then run Moran's I test (without reducing power of the test) for each individual features across all the suburbs considered for the metrics of M2 and M3 and note the $p-value$ (Note: This test takes in only one feature at a time). If $p<\alpha$ for a feature $F$, we conclude that $F$ has significant spatial autocorrelation (SA), i.e. geographic proximity has significant effect on $F$.

\subsubsection{M2}\leavevmode

Moran's I for Services Score: -0.00848

Moran's I p-value: 0.31300

\% of significant features (out of the all features for M2): 21\%

\% of feature-value of significant features: 15.25\%

(Out of the total values across all the suburbs and all the features. This works since all the values are non-negative.)

Procrustes Disparity between suburb $L^2$ distances and M2: 0.868

\hyperref[fig:moran_part_A]{Figure~\ref{fig:moran_part_A} Moran Statistics plot with p-values}


\subsubsection{M3}\leavevmode

\% of significant features (out of the all features for M3): 50\%

\% of feature-value of significant features: 77.90\%

Procrustes Disparity between suburb $L^2$ distances and M3: 0.747

\hyperref[fig:moran_part_A]{Figure~\ref{fig:moran_part_A} Moran Statistics plot with p-values}


\subsubsection{Conclusion}\leavevmode

The data partially supports the hypothesis, and depends on the perspective and the metric chosen. Some features being considered under the same metric may be significantly affected by physical proximity, while some may not be. Thus, we decided to consider the weight contributed by the feature values. Unsurprisingly, due to larger weight of M3 features as compared to M2, the disparity between the physical distance based proximity and M3 is much lower than that of disparity between M2 and physical distance.

\pagebreak

\section{Part B}

\subsection{B1.}

\subsubsection{Analysis focused on Hospitals}

We collected the actual locations of the hospitals from Google Maps and plotted them, in order to get an approximate idea of their positions \hyperref[fig:hosp_fig]{Figure~\ref{fig:hosp_fig}}.

To guide our analysis, we prepared followed the questions:
\begin{enumerate}
  \item[\textit{HQ1.}] \textit{Are hospital separations in proportion to the city population?}
  \item[\textit{HQ2.}] \textit{How do the population density vary as distance to the nearest hospital grows?}
  \item[\textit{HQ3.}] \textit{Do nearby hospitals with emergency departments to the populations experience a greater fraction of unnecessary emergency visits?}
\end{enumerate}

Note: The relevant model outputs for this part are given in \hyperref[tab:summary_SDM]{SDM Summary Table~\ref{tab:summary_SDM}}. All the relevant figures are contained within \hyperref[fig:hosp_fig]{Figure~\ref{fig:hosp_fig}}. Each of \textit{HQ1, HQ2, HQ3} has its respective output and plots given in them.



\textit{HQ1:} Visually the dependence looks quite linear. The notion is further strengthened by the Pearson correlation coefficient of 0.9130 and through simple linear regression with $R^2=0.8336$. Although the assumptions of linear regression are violated here, we apply anyway and check for non-independence within the model residuals $\epsilon_i$ (which must be independent for simple linear regression). 

Moran's I p-value for the OLS residuals (using same config as in Part A) is 0.0023 which implies autocorrelated error terms. Thus inferring anything from the p-values of OLS might lead to false conclusion even though the $R^2$ is high.

Thus, we resort to using the Spatial-Durbin Model\textsuperscript{\cite{atikah2021efficiency} \cite{Anselin1988} \cite{doi:10.1080/17421770601009841}} (SDM) which can handle both autocorrelated predictor and predicted variables.


\textit{HQ2:} We can observe an approximate linear decay in population density wrt the distances with correlation (Pearson)= -0.7364. The results of SDM are given in the said table.

\textit{HQ3:} Visually a decreasing trend can be observed. correlation (Pearson)$= -0.6869$. SDM results are given in the same table.


\subsubsection{Socio-Demographic Analysis}
Only the numeric features for the suburbs without any missing values were considered. Amongst them we narrowed down the features to only those with significant SA (Moran's I). Next, Pearson and Spearman correlation-coefficients was considered between every pair of spatially autocorrelated features and only the pairs with both correlation-coefficients  $>0.7$ or $<-0.7$ are considered for further analysis. We came down to 8 features in the said manner from a total of 56 features.

Note: All the plots and numeric outputs for \textit{SDQ1, SDQ2, SDQ3} (as specified below) are contained within \hyperref[tab:summary_SDM]{SDM Summary Table~\ref{tab:summary_SDM}} and \hyperref[fig:socio_demo_fig]{Figure~\ref{fig:socio_demo_fig}}

Afterwards we posed the following questions:

\begin{enumerate}
  \item[\textit{SDQ1.}] Does there exist a significant relation between the fraction of the population employed in the top occupation and the fraction of highly educated population?
  \item[\textit{SDQ2.}] How do the fraction of dwellings without internet vary with household incomes?
  \item[\textit{SDQ3.}] Is there a significant spatial pattern on a more local level regarding the lower level of education? 
\end{enumerate}

For both \textit{SDQ1} and \textit{SDQ2} we employed SDM for establishing relationships while for \textit{SDQ3} we used \textit{LISA}\textsuperscript{\cite{Anselin1995}, \cite{darribasLab_05}, \cite{geodacenterLocalSpatial}} to cluster the regions of lower educational level as per local spatial autocorrelation.

\subsection{B2}

\subsubsection{Hospital Data Analysis: Interpretation}\leavevmode

\textit{HQ1:} Hospital separations show a positive, statistically significant association with population size (\textit{x\_coef} = 0.268, \textit{p} = 0.0000), indicating that separations increase as population grows. Spatial autocorrelation (\textit{y\_lag} = 0.615, \textit{p} = 0.001) suggests high-separation areas tend to cluster together.

\textit{HQ2:} Population density decreases significantly with distance from the nearest hospital (\textit{x\_coef} = -242.925, \textit{p} = 0.00030), implying hospitals are often located near higher-density areas. Spatial autocorrelation (\textit{y\_lag} = 0.603, \textit{p} = 0.00179) shows neighboring areas share similar population density.

\textit{HQ3:} There is a negative relationship between non-urgent emergency visits and distance to the nearest hospital (\textit{x\_coef} = -0.418, \textit{p} = 0.0136), suggesting people closer to emergency departments use them more for non-urgent cases. A strong spatial lag (\textit{y\_lag} = 0.754, \textit{p} = 0.000) indicates neighboring regions exhibit similar patterns in emergency department use.


\subsubsection{Socio-Demographic Analysis: Interpretation}\leavevmode

\textit{SDQ1:} The percentage of people holding a degree or higher shows a positive, statistically significant association with the percentage of people in top occupations ($x_{\text{coef}} = 1.13616$, $p = 0.0000$), indicating that higher top occupation percentages are linked to higher educational attainment. The SA of ($y_{\text{lag}} = 0.553$, $p = 0.00729$) suggests that high educational attainment areas tend to cluster together.

\textit{SDQ2:} The percentage of households with income less than \$600 per week shows a positive, statistically significant association with the percentage of dwellings with no internet ($x_{\text{coef}} = 1.392$, $p = 0.0000$), indicating that areas with more dwellings lacking internet tend to have higher proportions of low-income households. The SA ($y_{\text{lag}} = 0.638$, $p = 0.00037$) suggests that low-income areas tend to cluster together.

\textit{SDQ3:} The percentage of individuals who did not complete year 12 shows a positive, statistically significant association with spatial clustering, with high values forming a High-High cluster (\text{mean} = 52.41\%, \text{std} = 7.30) and low values forming a Low-Low cluster (\text{mean} = 21.59\%, \text{std} = 4.86). This indicates that areas with higher rates of non-completion tend to cluster together, especially farther from the Melbourne GPO (\text{mean distance for High-High cluster} = 85.17 \ \text{km}, p < 0.05), while lower rates cluster closer to the center (\text{mean distance for Low-Low cluster} = 7.30 \ \text{km}, p < 0.05).



\subsection{B3}

\subsubsection{Reflection upon Hospital Data Analysis} \leavevmode
 
This analysis highlighted spatial dependencies that weren’t immediately visible. For instance, hospital separations increase with population size, and hospitals are often located near higher-density areas. We also saw that non-urgent emergency visits are more common closer to emergency departments, with similar patterns appearing in neighboring areas. These insights can help with decisions on hospital locations, emergency resource allocation, and adjusting healthcare services in denser areas, supporting better access to healthcare across Melbourne.

\subsubsection{Reflection upon Socio-Demographic Analysis}\leavevmode

In a similar way, the socio-demographic analysis revealed patterns like a link between education levels and employment in top occupations, an association between low-income households and internet access, and clusters of areas with lower educational attainment farther from Melbourne's center. These insights can guide initiatives like targeted job programs, investments in digital access for low-income neighborhoods, and support for communities with lower school completion rates, helping to address socioeconomic disparities.

\subsection{Personal Remarks}

In this project we used a variety of well-established techniques in spatial econometrics to gain insights that traditional ML models might struggle to capture. \href{https://en.wikipedia.org/wiki/Luc_Anselin}{Luc Anselin's} pioneering works\textsuperscript{\cite{Anselin1988}, \cite{Anselin1995}} has paved the way for regional sciences. His contributions continue to shape spatial analysis, enabling data-driven insights that support targeted, effective decision-making.



\pagebreak
\clearpage

\onecolumn

\begin{figure}[htbp]
  \vspace{-1cm}
  \centering
  \subfigure[Joint Density of Suburb Coordinates (Relative)]{
      \includegraphics[scale=0.5]{figures/joint_density.png}
      \label{fig:jd_suburb_dist}
  }
  \hspace{0cm} % Adjust horizontal spacing
  \subfigure[Scatterplot of the Suburbs]{
      \includegraphics[scale=0.3]{suburbs_cartesian.png}
      \label{fig:scatter_suburb}
  }
  
  \vspace{0cm} % Adjust vertical spacing between rows of figures
  
  \subfigure[MDS of Services]{
      \includegraphics[scale=0.5]{mds_services.png}
      \label{fig:mds_services}
  }
  \subfigure[MDS Population Fraction]{
    \includegraphics[scale=0.5]{MDS_Population_Fraction.png}
    \label{fig:mds_population}
}
  \hspace{-0.4cm}
  \subfigure[Spectral Clustering]{
      \includegraphics[scale=0.42]{spectral_clustering.png}
      \label{fig:spectral_clustering}
  }
\end{figure}


\begin{figure}
  \vspace{-0.6cm}
  \includegraphics[scale=0.5]{moran_i_analysis.png}
  \caption{Moran's I analysis for the features}
  \label{fig:moran_part_A}
\end{figure}





\begin{figure}
  \centering
  \includegraphics[scale=0.5]{hosp_full_with_labels.png}
  \caption{Hospital Data Analysis Plots}
  \label{fig:hosp_fig}
\end{figure}


\pagebreak
\clearpage

\begin{figure}
  \centering
  \includegraphics[scale=0.45]{socio_demo_comb.png}
  \caption{Socio-Demographic Analysis Plots}
  \label{fig:socio_demo_fig}
\end{figure}

\hspace{-1cm}
\begin{table}[h!]
\fontsize{8}{6}\selectfont % Set font size for the table
\setlength{\tabcolsep}{2pt} % Reduce the space between columns
\renewcommand{\arraystretch}{1.5} % Increase the row height
\begin{tabular}{|p{2cm}|p{2cm}|p{1.5cm}|p{1.5cm}|p{1cm}|p{1cm}|p{1cm}|p{1cm}|p{1cm}|p{1cm}|p{1cm}|p{1cm}|p{1cm}|p{1cm}|} % Add vertical bars between columns with wrapping
\hline
y & X & const & const\_pval & x\_coef & x\_pval & y\_lag & y\_pval & x\_lag & x\_pval & pR\^{}2 & d\_imp & i\_imp & t\_imp \\
\hline
Public hospital separations, 2012-13 & 2012 ERP, total & 369.292 & 0.714 & 0.268 & 0.000 & 0.615 & 0.001 & -0.200 & 0.030 & 0.866 & 0.268 & -0.091 & 0.177 \\
\hline
Population Density & Distance to nearest public hospital & 2176.540 & 0.039 & -242.93 & 0.000 & 0.603 & 0.002 & 7.152 & 0.964 & 0.655 & -242.93 & -351.51 & -594.43 \\
\hline
Category 4 \& 5 emergency department presentations, \% & Distance to nearest public hospital with emergency department & 17.092 & 0.036 & -0.418 & 0.014 & 0.754 & 0.000 & -0.331 & 0.419 & 0.610 & -0.418 & -2.628 & -3.046 \\
\hline
Holds degree or higher, \% & Top occupation, \% & -13.875 & 0.016 & 1.136 & 0.000 & 0.553 & 0.007 & 0.025 & 0.957 & 0.899 & 1.136 & 1.461 & 2.597 \\
\hline
Equivalent household income \$600/week, \% & Dwellings with no internet, \% & 2.979 & 0.641 & 1.392 & 0.000 & 0.638 & 0.000 & -0.682 & 0.201 & 0.729 & 1.392 & 0.570 & 1.962 \\
\hline
\end{tabular}
\caption{Summary of SDMs}
\label{tab:summary_SDM}
\end{table}


\begin{small}
  Column ID Abbreviations:
  
  y: Dependent Variable, X: Predictor Variable, const: Constant Coefficient, const\_pval: Constant p-value, x\_coef: Predictor Coefficient, x\_pval: Predictor p-value, y\_lag: Dependent Variable Lag, y\_pval: p-value for Dependent Lag, x\_lag: Predictor Variable Lag, x\_pval: p-value for Predictor Lag, pR\^{}2: Pseudo R-squared, d\_imp: Direct Impact, i\_imp: Indirect Impact, t\_imp: Total Impact.
\end{small}

\begin{table}[h!]
    \centering
    \begin{tabular}{|c|l|c|l|}
    \hline
    \textbf{Index} & \textbf{Community Name} & \textbf{Index} & \textbf{Community Name} \\ \hline
    0  & Ascot Vale          & 17 & Parkville           \\
    1  & Braybrook           & 18 & Pascoe Vale South   \\
    2  & Craigieburn         & 19 & Port Melbourne      \\
    3  & Croydon             & 20 & Prahran             \\
    4  & Fawkner             & 21 & Somerville          \\
    5  & Footscray           & 22 & Sorrento            \\
    6  & Glenroy             & 23 & South Melbourne     \\
    7  & Malvern East        & 24 & South Yarra         \\
    8  & Malvern             & 25 & Springvale          \\
    9  & Melbourne Airport   & 26 & St Andrews Beach    \\
    10 & Mentone             & 27 & St Kilda East       \\
    11 & Moorabbin           & 28 & St Kilda            \\
    12 & Mordialloc          & 29 & St Kilda West       \\
    13 & Murrumbeena         & 30 & Toorak              \\
    14 & Noble Park          & 31 & Tyabb               \\
    15 & North Melbourne     & 32 & Waterways           \\
    16 & Northcote           & 33 & Windsor             \\ \hline
    \end{tabular}
    \caption{Index of Communities}
    \label{tab:merged_community_names_serial}
    \end{table}
    



% \begin{table}[ht]
%   \centering
%   \begin{tabular}{|c|l|}
%   \hline
%   \textbf{No.} & \textbf{Feature} \\ \hline
%   0 & Dwellings with no internet, \% \\ \hline
%   1 & Personal income \$<400/week, \% \\ \hline
%   2 & IRSD (avg) \\ \hline
%   3 & Equivalent household income \$<600/week, \% \\ \hline
%   4 & Did not complete year 12, \% \\ \hline
%   5 & Holds degree or higher, \% \\ \hline
%   6 & Top occupation, \% \\ \hline
%   7 & IRSD (max) \\ \hline
%   \end{tabular}
%   \caption{List of Features for Correlation Plot}
%   \end{table}
  

\twocolumn
\printbibliography

\end{document}