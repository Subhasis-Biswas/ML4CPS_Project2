%%%%%%%%%%%%%%%%%%%%%%%%%%%%%%%%%%%%%%%%%
% Journal Article
% LaTeX Template
% Version 2.0 (February 7, 2023)
%
% This template originates from:
% https://www.LaTeXTemplates.com
%
% Author:
% Vel (vel@latextemplates.com)
%
% License:
% CC BY-NC-SA 4.0 (https://creativecommons.org/licenses/by-nc-sa/4.0/)
%
% NOTE: The bibliography needs to be compiled using the biber engine.
%
%%%%%%%%%%%%%%%%%%%%%%%%%%%%%%%%%%%%%%%%%

%----------------------------------------------------------------------------------------
%	PACKAGES AND OTHER DOCUMENT CONFIGURATIONS
%----------------------------------------------------------------------------------------

\documentclass[
	a4paper, % Paper size, use either a4paper or letterpaper
	10pt, % Default font size, can also use 11pt or 12pt, although this is not recommended
	unnumberedsections, % Comment to enable section numbering
	twoside, % Two side traditional mode where headers and footers change between odd and even pages, comment this option to make them fixed
]{LTJournalArticle}

\addbibresource{ref.bib} % BibLaTeX bibliography file


\setcounter{page}{1} % The page number of the first page, set this to a higher number if the article is to be part of an issue or larger work

\usepackage{url}

\usepackage{graphicx}             % For including images
\usepackage{adjustbox}            % For scaling and packing
\usepackage{subcaption}           % For subfigures
\usepackage{xcolor}         % For color names
\usepackage{hyperref}       % For hyperlinks and colored citations

%%%%

\usepackage{amssymb,amsthm,amsmath}
\usepackage{multirow,epsfig, times,multicol}
\usepackage[mathscr]{eucal}
\usepackage{amsfonts}
\usepackage{wrapfig}
\usepackage{amssymb}

\usepackage{ifthen}

%%%%

% Set colors for links and citations
\hypersetup{
    colorlinks=true,        % Enable colored links
    linkcolor=gray,        % Set link color (e.g., section refs)
    citecolor=blue,         % Set citation color
    urlcolor=cyan           % Set URL color
}



\usepackage{listings}


\lstset{
    basicstyle=\ttfamily, % Use typewriter font
    keywordstyle=\color{blue}, % Keywords in blue
    commentstyle=\color{green!50!black}, % Comments in green
    stringstyle=\color{red}, % Strings in red
    backgroundcolor=\color{gray!10}, % Light gray background
    numbers=left, % Line numbers on the left
    numberstyle=\tiny\color{gray}, % Style for line numbers
    frame=single, % Single frame around code
    tabsize=2, % Set tab size
}

%----------------------------------------------------------------------------------------
%	TITLE SECTION
%----------------------------------------------------------------------------------------

\title{CP 217: Project 2 Report} % Article title, use manual lines breaks (\\) to beautify the layout

% Authors are listed in a comma-separated list with superscript numbers indicating affiliations
% \thanks{} is used for any text that should be placed in a footnote on the first page, such as the corresponding author's email, journal acceptance dates, a copyright/license notice, keywords, etc
\author{%
	Subhasis Biswas\thanks{22571, CDS MTech 2nd Year},  Pradhumn Sharma\thanks{22559, CDS MTech 2nd Year} and Bhookya Raju\thanks{25076, MTech Mobility Engg \& Mech Dept}
}

% Affiliations are output in the \date{} command


%----------------------------------------------------------------------------------------

\begin{document}

\maketitle % Output the title section


\section{Part A}



\subsection{A.1}

In this section we refer to the defined metrics as $M1$, $M2$ and $M3$.
\begin{itemize}
    \item \textbf{M1}: Geographic Proximity
    \item \textbf{M2}: Service Availability
    \item \textbf{M3}: Population Age Distribution
\end{itemize}

\subsubsection{Metric Specifications}\leavevmode


Here we'll be going through how the similarity metrics were defined and the related information.

\textbf{M1}: Used $Location$ feature from the excel sheets to approximate the location relative to the \textit{Melbourne General Post Office (GPO)}. The feature-value was parsed as: \newline
\textit{Xkm Y of Melbourne} $\rightarrow (X, Y) \rightarrow (X, compass\ bearing)$, with $Y \in \{\text{N}, \text{NNE}, \text{NE}, \text{ENE}, \dots,\text{NNW}\}$ and the corresponding bearings being in $\{0, 22.5, 45, 67.5,\dots, 337.5\}$.

Now, max distance of any suburb from our origin $O$ (Melbourne GPO) is 68km. So, maximum pairwise distance that can be is $2\times 68=136$ km. The curvature of the Earth causes a drop of around $1.45km$ [\href{https://earthcurvature.com/}{Calculator}] across this distance. As this is quite small, it's ignored and the entire region is treated as flat. We then convert $Polar$ to $Cartesian$ using the formula 

\[
x = r \cdot \sin\left(\frac{\pi}{180} \cdot \theta\right)
\]
\[
y = r \cdot \cos\left(\frac{\pi}{180} \cdot \theta\right)
\] [$y$ axis aligns with $0^\circ$, so $\sin$ and $\cos$ are swapped]. The goal is to check for physical proximity of the suburbs, using the $L^2$ norm.

\textbf{M2}: Each feature under the $Services$ catgory is considered and min-max scaled. Before Scaling StDev <Figure> and After Scaling StDev <Figure>. Afterwards, mean was taken across the scaled feature-values under the Services category per row (suburb by suburb), which is named as the $Services \ Score$ for that suburb. The distribution of is as such <Figure>. Similarity between suburbs from this perspective is defined as the absolute difference between the corresponding services' score. This is the pairwise similarity heatmap <Figure>. The aim is to quantify how well-equipped a suburb is in terms of the said facilities and to check for similar amount of services provided in other suburbs. However, some feature values like \textit{Aged Care (Low Care)} dominate over the others. Thus, scaling is required. 

\textbf{M3}: The \textit{2012 population} category was chosen and only the feature names with $\%$ were chosen, for example

$\{ \textit{2012 ERP age 0-4\%}, \textit{2012 ERP age 5-9\%}, \dots\}$. The feature values are then converted to probability distributions with range $[0,1]$, with each row summing up to 1 for each suburb (verified). For quantifying similarity between two locations we consider the \href{https://en.wikipedia.org/wiki/Jensen%E2%80%93Shannon_divergence}{Jensen Shannon Divergence}(JSD) between the two population distributions. This distance metric was chosen as we're essentially dealing with pmf's and because JSD is symmetric. The aim of this metric/perspective is to measure the similarity between the 2012 (latest) population distributions between two locations.



\subsection{A.2}

We are using the following module for computing the multidimensional scaling onto 2D.
\begin{verbatim}
    from sklearn.manifold import MDS
\end{verbatim}

We'll be using \href{https://en.wikipedia.org/wiki/Procrustes_analysis}{Procrustes Disparity} for measuring the alignment between two similarity/distance metrics, the comparisons might be of the sort: \textit{Metric in original space vs Metric in 2D} or \textit{Metric A vs Metric B} etc. The closer the value of the disparity is to 0 the better. For MDS, we'll also note the \href{https://imaging.mrc-cbu.cam.ac.uk/statswiki/FAQ/mds/stress#:~:text=The%20measure%20of%20goodness%20of,or%20more%20estimated%20stimuli%20dimensions}{stress of the MDS} and the correlation between the original pairwise distances and distances in the 2D projected space. 

\textbf{M1}: The suburb locations are already on the 2D map, thus we do not require MDS. 

<Figure Placeholder, Plain Scatterplot> <Figure Placeholder, Geo-accurate> <Figure Placeholder Joint Density>

\textbf{M2}: 2D projected profile: <Figure>

\textbf{M3}: 2D projected profile: <Figure>





\subsubsection{M3}
<MDS Pop Distribution; 2D corr MDS; Stress MDS>

\subsubsection{Findings}
<Similar for population; Expected; Procrus> 

<Diff for service; unexpected; procus>

<spectral cluster using distance matrix; then find common stuff; declare that geographic sim is referred in A3>


\subsection{A.3}


<Morans for M2; Weight fraction analysis; Plots>
<M2: Weight less + procrus + not good>


<Morans for M3; Weight fraction analysis; Plots>
<M3: Weight More + procrus + good>



---- 

Explain why 











\printbibliography

\end{document}