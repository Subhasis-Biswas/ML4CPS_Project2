%%%%%%%%%%%%%%%%%%%%%%%%%%%%%%%%%%%%%%%%%
% Journal Article
% LaTeX Template
% Version 2.0 (February 7, 2023)
%
% This template originates from:
% https://www.LaTeXTemplates.com
%
% Author:
% Vel (vel@latextemplates.com)
%
% License:
% CC BY-NC-SA 4.0 (https://creativecommons.org/licenses/by-nc-sa/4.0/)
%
% NOTE: The bibliography needs to be compiled using the biber engine.
%
%%%%%%%%%%%%%%%%%%%%%%%%%%%%%%%%%%%%%%%%%

%----------------------------------------------------------------------------------------
%	PACKAGES AND OTHER DOCUMENT CONFIGURATIONS
%----------------------------------------------------------------------------------------

\documentclass[
	a4paper, % Paper size, use either a4paper or letterpaper
	10pt, % Default font size, can also use 11pt or 12pt, although this is not recommended
	unnumberedsections, % Comment to enable section numbering
	twoside, % Two side traditional mode where headers and footers change between odd and even pages, comment this option to make them fixed
]{LTJournalArticle}

\addbibresource{ref.bib} % BibLaTeX bibliography file


\setcounter{page}{1} % The page number of the first page, set this to a higher number if the article is to be part of an issue or larger work

\usepackage{url}

\usepackage{graphicx}             % For including images
\usepackage{adjustbox}            % For scaling and packing
\usepackage{subcaption}           % For subfigures
\usepackage{xcolor}         % For color names
\usepackage{hyperref}       % For hyperlinks and colored citations

%%%%

\usepackage{amssymb,amsthm,amsmath}
\usepackage{multirow,epsfig, times,multicol}
\usepackage[mathscr]{eucal}
\usepackage{amsfonts}
\usepackage{wrapfig}
\usepackage{amssymb}

\usepackage{ifthen}


\usepackage{titlesec}

\titlespacing*{\section} % Adjusts section spacing
  {0pt}   % Left indent (usually 0pt for no indent)
  {2pt}  % Space above the section title
  {2pt}   % Space below the section title

\titlespacing*{\subsection}
  {0pt}{2pt}{2pt}


%%%%

% Set colors for links and citations
\hypersetup{
    colorlinks=true,        % Enable colored links
    linkcolor=gray,        % Set link color (e.g., section refs)
    citecolor=blue,         % Set citation color
    urlcolor=cyan           % Set URL color
}



\usepackage{listings}


\lstset{
    basicstyle=\ttfamily, % Use typewriter font
    keywordstyle=\color{blue}, % Keywords in blue
    commentstyle=\color{green!50!black}, % Comments in green
    stringstyle=\color{red}, % Strings in red
    backgroundcolor=\color{gray!10}, % Light gray background
    numbers=left, % Line numbers on the left
    numberstyle=\tiny\color{gray}, % Style for line numbers
    frame=single, % Single frame around code
    tabsize=2, % Set tab size
}

\usepackage{enumitem} % Recommended package for list customization

\setlist[itemize]{topsep=1pt, itemsep=0pt, parsep=1pt, partopsep=0pt}
\setlength{\topsep}{0pt}     % Space before and after the environment
\setlength{\partopsep}{0pt}   % Extra space added before if needed
\setlength{\parskip}{0pt}     % Space between paragraphs inside verbatim (optional)



%----------------------------------------------------------------------------------------
%	TITLE SECTION
%----------------------------------------------------------------------------------------

\title{CP 217: Project 2 Report} % Article title, use manual lines breaks (\\) to beautify the layout

% Authors are listed in a comma-separated list with superscript numbers indicating affiliations
% \thanks{} is used for any text that should be placed in a footnote on the first page, such as the corresponding author's email, journal acceptance dates, a copyright/license notice, keywords, etc
\author{%
	Subhasis Biswas\thanks{22571, CDS MTech 2nd Year},  Pradhumn Sharma\thanks{22559, CDS MTech 2nd Year} and Bhookya Raju\thanks{25076, MTech Mobility Engg \& Mech Dept}
}

% Affiliations are output in the \date{} command


%----------------------------------------------------------------------------------------

\begin{document}

% \onecolumn
% \maketitle % Output the title section



% \thispagestyle{empty}
% \mbox{} % Empty box to force the page

% \newpage

\section{Part A}



\subsection{A.1}

In this section we refer to the defined metrics as $M1$, $M2$ and $M3$.
\begin{itemize}
    \item \textbf{M1}: Geographic Proximity
    \item \textbf{M2}: Service Availability
    \item \textbf{M3}: Population Age Distribution
\end{itemize}

\subsubsection{Metric Specifications}\leavevmode


Here we'll be going through how the similarity metrics were defined and the related information.

\textbf{M1}: Used $Location$ feature from the excel sheets to approximate the location relative to the \textit{Melbourne General Post Office (GPO)}. The feature-value was parsed as: \newline
\textit{Xkm Y of Melbourne} $\rightarrow (X, Y) \rightarrow (X, compass\ bearing)$, with $Y \in \{\text{N}, \text{NNE}, \text{NE}, \text{ENE}, \dots,\text{NNW}\}$ and the corresponding bearings being in $\{0, 22.5, 45, 67.5,\dots, 337.5\}$.

Now, max distance of any suburb from our origin $O$ (Melbourne GPO) is 68km. So, maximum pairwise distance that can be is $2\times 68=136$ km. The curvature of the Earth causes a drop of around $1.45km$ [\href{https://earthcurvature.com/}{Calculator}] across this distance. As this is quite small, it's ignored and the entire region is treated as flat. We then convert $Polar$ to $Cartesian$ using the formula 

\[
x = r \cdot \sin\left(\frac{\pi}{180} \cdot \theta\right)
\]
\[
y = r \cdot \cos\left(\frac{\pi}{180} \cdot \theta\right)
\] [$y$ axis aligns with $0^\circ$, so $\sin$ and $\cos$ are swapped]. The goal is to check for physical proximity of the suburbs, using the $L^2$ norm.

\textbf{M2}: Each feature under the $Services$ catgory is considered and min-max scaled. Before Scaling StDev <Figure> and After Scaling StDev <Figure>. Afterwards, mean was taken across the scaled feature-values under the Services category per row (suburb by suburb), which is named as the $Services \ Score$ for that suburb. The distribution of is as such <Figure>. Similarity between suburbs from this perspective is defined as the absolute difference between the corresponding services' score. This is the pairwise similarity heatmap <Figure>. The aim is to quantify how well-equipped a suburb is in terms of the said facilities and to check for similar amount of services provided in other suburbs. However, some feature values like \textit{Aged Care (Low Care)} dominate over the others. Thus, scaling is required. 

\textbf{M3}: The \textit{2012 population} category was chosen and only the feature names with $\%$ were chosen, for example

$\{ \textit{2012 ERP age 0-4\%}, \textit{2012 ERP age 5-9\%}, \dots\}$. The feature values are then converted to probability distributions with range $[0,1]$, with each row summing up to 1 for each suburb (verified). For quantifying similarity between two locations we consider the \href{https://en.wikipedia.org/wiki/Jensen%E2%80%93Shannon_divergence}{Jensen Shannon Divergence}(JSD) between the two population distributions. This distance metric was chosen as we're essentially dealing with pmf's and because JSD is symmetric. The aim of this metric/perspective is to measure the similarity between the 2012 (latest) population distributions between two locations.



\subsection{A.2}

We are using the following module for computing the multidimensional scaling onto 2D.
\begin{verbatim}
    from sklearn.manifold import MDS
\end{verbatim}

We'll be using \href{https://en.wikipedia.org/wiki/Procrustes_analysis}{Procrustes Disparity} for measuring the alignment between two similarity/distance metrics, the comparisons might be of the sort: \textit{Metric in original space vs Metric in 2D} or \textit{Metric A vs Metric B} etc. The closer the value of the disparity is to 0 the better. For MDS, we'll also note the \href{https://imaging.mrc-cbu.cam.ac.uk/statswiki/FAQ/mds/stress#:~:text=The%20measure%20of%20goodness%20of,or%20more%20estimated%20stimuli%20dimensions}{stress of the MDS} and the correlation between the original pairwise distances and distances in the 2D projected space. The values of the MDS quality metrics are written within the plots.

\textbf{M1}: The suburb locations are already on the 2D map, thus we do not require MDS. 

<Figure Placeholder, Plain Scatterplot> <Figure Placeholder, Geo-accurate> <Figure Placeholder Joint Density>

\textbf{M2}: 2D projected profile: <Figure>

\textbf{M3}: 2D projected profile: <Figure>


\subsubsection{Findings}

While doing MDS, for the service score metric, we're essentially dealing with scalar outputs, and trying to put them in 2D. Thus the distance has been well preserved between the original and the 2D space as indicated by the low stress and disparity as well as high correlation. This was expected.

However, for JSD based population distribution similarity, the high amount of stress indicates poor projection onto the 2D. Although it's not exactly a surprise but still that's quite a high amount. Despite of that, the disparity is still low and correlation is high. 


To check which suburbs remain unchanged between these two metrics \textit{M2} and \textit{M3} (we deal with geographic proximity later on) we do a \href{https://en.wikipedia.org/wiki/Spectral_clustering}{spectral clustering} with the exact same similarity matrices arising out of the original computation. Then we use a Gaussian Kernel with $\sigma=1$. We check for the 'elbow' in the (sorted in descending order) eigenvalue plots for both M2 and M3 for the affinity matrices and decide upon the number of clusters required. We settle on $k=3$. 

We then spectral-cluster the suburbs and the labels for the two suburbs are then matched (since labels for M2 may be mapped to different labels of M3) using the \href{https://en.wikipedia.org/wiki/Hungarian_algorithm}{Hungarian Algorithm} for optimal label assignment between the cluster labels arising out of M2 and M3. 

After matching the labels, $16$ (47\%) suburbs got the same labels, and the indices are given <here>.



\subsection{A.3}


The issue with geospatial data analysis is that the observations of the neighbouring places are not independent, thus many of the ML algorithms that assume independence of the samples cannot be used here directly. Here, in order to address the \href{https://www.sciencedirect.com/topics/mathematics/spatial-autocorrelation}{Spatial Autocorrelation} we use \href{https://en.wikipedia.org/wiki/Moran%27s_I}{Moran's I} (both Global and Local, with permutations set to 9999, random seed = 42) for testing the influence of the geographic proximity. More formally, for Moran's I hypothesis test we set the null hypothesis $H_0$ as:

$H_0:$ \textit{There is no spatial autocorrelation between the observations}

with significance level set to $\alpha =0.05$. Weight matrix was constructed using KNN with $k=8$ (\href{https://pro.arcgis.com/en/pro-app/latest/tool-reference/spatial-statistics/spatial-autocorrelation.htm}{Recommendation}) due to the varying spatial density of the suburbs.

We then run Moran's I test for each individual features across all the suburbs considered for the metrics of M2 and M3 and note the $p-value$ (Note: This test takes in only one feature at a time). If $p<\alpha$ for a feature $F$, we conclude that $F$ has significant spatial autocorrelation, i.e. geographic proximity has significant effect on $F$.

\subsubsection{M2}\leavevmode

Moran's I for Services Score: -0.00848

Moran's I p-value: 0.31300

\% of significant features (out of the all features for M2): 21\%

\% of feature-value of significant features: 15.25\%

(Out of the total values across all the suburbs and all the features. This works since all the values are non-negative.)

Procrustes Disparity between suburb $L^2$ distances and M2: 0.868

<Table for Moran's I significant>

<Figure for MI for M2>

<Local Moran Figures>



\subsubsection{M3}\leavevmode

\% of significant features (out of the all features for M3): 50\%

\% of feature-value of significant features: 77.90\%

Procrustes Disparity between suburb $L^2$ distances and M3: 0.747

<Table for Moran's I significant>

<Figure for MI for M3>

<Local Moran Figures>

---- 

\subsubsection{Conclusion}\leavevmode

The data partially supports the hypothesis, and depends on the perspective and the metric chosen. Some features being considered under the same metric may be significantly affected by physical proximity, while some may not be. Thus, we decided to consider the weight contributed by the feature values. Unsurprisingly, due to larger weight of M3 features as compared to M2, the disparity between the physical distance based proximity and M3 is much lower than that of disparity between M2 and physical distance.

\section{Part B}


\subsection{Hospital Analysis Outline}

<Public hospital separation plot>

We collected the actual locations of the hospitals from Google Maps and plotted them, in order to get an idea of their positions. <Figure>

We do a distribution plot for 'Distance to nearest public hospital (km)'. <Figure>

This are the plots depicting the which hospitals (with and without emergency) serve how much of the nearest population. <Fig1> <Fig2>

These are the tables for the same <Tab1> <Tab2>

We bin the distances according to Freedman-Diaconis rule. The formula:

$\text{Bin Width} = 2 \times \frac{\text{IQR}}{\sqrt[3]{n}}$




We check if the hospital separations of each of the suburbs are linearly related to the respective total populations of 2012. The initial guess is that they should be. <Figure>

Correlation (Pearson): 0.9130

$R^2$ for Linear Regression: 0.8336

As can be seen, a lot of the separations can be explained by a linear relationship between the two variables.

Now, let us check if population density has anything to do with the distance to the nearest hospital. The initial guess is that due to the lesser availability of essential services, with increasing distance, the density should also decay. We log-transform the travel-time. <Figure>

We can observe a linear decay in population density wrt $\log(distance \ to \ nearest \ public \ hospital)$. We try a least-square fit of a linear model. 

Correlation (Pearson): -0.7364

$R^2: 0.5422$

Next up, we can ask, whether non-essential (Category 4 and 5) emergency visits decrease with distances to the hospitals with emergency services.

The boxplot <Figure> indeed shows a decreasing trend as the distances grow. The intervals are considered according to the said rule.


Another aspect is that the injuries can be spatially clustered by LISA. Disregarding the significance, we can plot the means of the distance to GPO for each cluster vs means of \% of injuries as emergency. We can clearly see an increasing trend as the distance from the Melbourne GPO increases, i.e. outskirts have more fraction of emergency as injuries.



MENTION MORAN's I VALUES FOR ALL.

\subsection{Population Increase/Decrease trend with distance}







\printbibliography

\end{document}